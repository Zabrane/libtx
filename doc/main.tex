\documentclass[10pt,letterpaper]{article}
\usepackage[english]{babel}
\usepackage[utf8x]{inputenc}
\usepackage[T1]{fontenc}
\usepackage[a4paper,top=3cm,bottom=2cm,left=3cm,right=3cm,marginparwidth=1.75cm]{geometry}
\usepackage{amsmath}
\usepackage{amssymb}
\usepackage{amsthm}
\usepackage{dsfont}
\usepackage{graphicx}
\usepackage{tikz}
\usepackage{indentfirst}

\newcommand \prop{\mathds{P}}
\newcommand \dom[1]{\text{dom}\left(#1\right)}
\newcommand \cod[1]{\text{cod}\left(#1\right)}
\newcommand \crpred[3]{\mathrel{\kappa\left(#1\,#2\,#3\right)}}
\newcommand \chain[3]{\mathrel{#1\stackrel{#3}{\rightsquigarrow}#2}}
\newcommand \bm[1]{\boldsymbol{#1}}
\newcommand \comm[2]{\mathrel{\text{comm}\left(#1,#2\right)}}
\newcommand \possible[1]{\mathrel{\text{possible}\left({#1}\right)}}
\newcommand \local[2]{\mathrel{\underset{#1}{\text{local}}\left(#2\right)}}
\newcommand \crlocal[2]{\mathrel{\underset{#1}{\text{chain\_rule\_local}}\left(#2\right)}}
\newcommand \mdoubleplus{\mathbin{+\mkern-6mu+}}
\newtheorem{theorem}{Theorem}
\theoremstyle{definition}
\newtheorem{definition}{Definition}
%% Title
\title{
		\huge Separation logic of traces (SLOT)
}
\selectlanguage{english}

\begin{document}
\maketitle

\begin{abstract}
  SLOT is an experimental framework that allows to formally prove
  temporal properties of functional programs with side effects. It is
  based entirely on constructive logic without additional axioms. It
  comes with a shallow embedded DSL for describing systems in Gallina
  language, and an extendable model of non-deterministic side effect
  handlers.
\end{abstract}

\section{Motivation}

\section{Supporting definitions}

SLOT is based on constructive logic rather than set theory, and relies
on Coq definition of ensemble. Ensemble $E$ on type $T$ is defined by
a predicate of type $T \to \prop$. Hypothesis that element $t$ belongs
to $E$ is defined as following: $t \in E \triangleq E\, t$. Ensemble
$E'$ is included in $E$ when
$E' \subset E \triangleq \forall t, t \in E' \to t \in E$.

\section{Core definitions}

Model of the system is defined by state space $S$, set of side effects
$T$, and predicate $\mathrel{\kappa:S\to S\to T\to \prop}$, called
chain rule. Hypothesis of type $\crpred{s}{s'}{t}$ provides evidence that
system can be transitioned from state $s$ to state $s'$ via side
effect $t$. Trace is a list of elements of $T$.

\begin{definition}
  Let $s$ and $s'$ are points of state space, and $\bm{t}$ is a trace.
  Long step property of a trace $\chain{s}{s'}{\bm{t}}$ is defined by
  the following rules:

  \begin{equation}
    \frac{s : S}{\chain{s}{s}{[]}} \text{ls\_nil}
  \end{equation}
  \\
  \begin{equation}
    \frac{s\, s' \, s'' : S \quad  t : T \quad \bm{t} : \text{list}\,T \quad \crpred{s}{s'}{t} \quad \chain{s'}{s''}{\bm{t}}}
         {\chain{s}{s''}{t::\bm{t}}}
    \text{ls\_cons}
  \end{equation}
\end{definition}

\begin{definition}
  Domain $\dom{\bm{t}}$ and codomain $\cod{\bm{t}}$ of a trace
  $\bm{t}$ are ensembles that satisfy the following property:
  \begin{equation}
    \forall s\, s', s \in \dom{\bm{t}} \land s' \in \cod{\bm{t}} \iff
    \chain{s}{s'}{t}
  \end{equation}
\end{definition}

\begin{definition}
  Trace $\bm{t}$ is called possible when its domain and codomain are
  inhibited. Or more formally:

  \begin{equation}
    \possible{\bm{t}} \triangleq \exists s\, s', \chain{s}{s'}{\bm{t}}
  \end{equation}
\end{definition}

Important to note: $\neg \possible{\bm{t}}$ hypothesis provides
evidence that sequence of side effects $\bm{t}$ cannot be produced by
a running system. This can be used to prove temporal safety
properties. Converse is not true: $\possible{\bm{t}}$ does not
guarantee that $\bm{t}$ can be produced by a running system, because
domain of $\bm{t}$ does not necessarily match with the set of valid
initial states of the system.

\begin{definition}
  Side effects $t_1$ and $t_2$ commute when

  \begin{equation}
  \comm{t_1}{t_2} \triangleq \forall s\,s', \chain{s}{s'}{[t_1,t_2]} \iff \chain{s}{s'}{[t_2,t_1]}
  \end{equation}

  \begin{figure}[h]
    \centering
    \begin{tikzpicture}[node distance=3cm, auto]
      \node (s) {$s$};
      \node (s1) [right of=s] {$s_1$};
      \node (s2) [below of=s] {$s_2$};
      \node (s') [right of=s2] {$s'$};
      \draw[->] (s) to node {$t_1$} (s1);
      \draw[->] (s1) to node {$t_2$} (s');
      \draw[->] (s) to node {$t_2$} (s2);
      \draw[->] (s2) to node {$t_1$} (s');
    \end{tikzpicture}
    \caption{Paths in state space implied by commutativity of $t_1$ and $t_2$}
    \label{fig:comm}
  \end{figure}
\end{definition}

\begin{definition}
  Let $P$ is a predicate called precondition, $Q$ is a predicate
  called postcondition and $\bm{t}$ is a trace. Hoare logic of traces
  consists of just one rule:

  \begin{equation}
    \frac{P\, Q: S \to \prop \quad \bm{t} \quad \forall s\,s', \chain{s}{s'}{\bm{t}} \to P s \to Q s'}
         {\{P\} \bm{t} \{Q\}}
    \text{hoare}
  \end{equation}

  This means that if initial state of the system belongs to the
  intersection of $\dom{\bm{t}}$ and precondition $P$ (dark gray area
  on fig.\ref{fig:hoare}), then every possible execution of trace
  $\bm{t}$ will leave system state on the intersection of
  $\cod{\fm{t}}$ and $Q$ (light gray area).

  \def\grdom{(0,0) circle (1.5cm)}
  \def\grcod{(0:5cm) circle (1.5cm)}
  \def\grpre{(1cm,2cm) circle (1.5cm)}
  \def\grpost{(6cm,2cm) circle (1.5cm)}

  \begin{figure}[h]
    \centering
    \begin{tikzpicture}
      \draw \grdom node[below] {$\dom{\bm{t}}$};
      \draw \grcod node [below] {$\cod{\bm{t}}$};
      \draw \grpre node [above] {$P$};
      \draw \grpost node [above] {$Q$};

      \begin{scope}
        \clip \grdom;
        \fill[gray] \grpre;
      \end{scope}
      \begin{scope}
        \clip \grpost;
        \clip \grcod;
        \fill[lightgray] \grpost;
      \end{scope}

      \draw [->] (0.0cm,1.1cm) -- (1.7cm, 1.6cm) -- (4.3cm, 1.5cm) -- (5.1cm, 1.2cm)
      \draw [->] (0.7cm,1.0cm) -- (2.2cm, 1.2cm) -- (4cm, 0.5cm) -- (5.3cm, 1.0cm)
      \draw [->] (0.7cm,1.0cm) -- (1.9cm, -0cm) -- (4.2cm, 0.01cm) -- (6cm, 0.7cm)
    \end{tikzpicture}
    \caption{Intuition behind $\{P\}\bm{t}\{Q\}$}
    \label{fig:hoare}
  \end{figure}
\end{definition}

\begin{theorem}
  Let $P$, $Q$, $R$ are asserts, $\bm{t_1}$ and $\bm{t_2}$ are traces,
  then from $\{P\}\bm{t_1}\{Q\}$ and $\{Q\}\bm{t_2}\{R\}$ follows
  $\{P\}\bm{t_1} \mdoubleplus \bm{t_2} \{R\}$.
\end{theorem}
\begin{proof}
See \mbox{hoare\_concat}.
\end{proof}

\begin{definition}
  Assert $P$ is called local under ensemble of side effects $E$ when

  \begin{equation}
  \local{E}{P} \triangleq \forall t:T,t \notin E \to \{P\} [t] \{P\}
  \end{equation}
\end{definition}

\begin{theorem}
  Let $E_1$ and $E_2$ are disjoint ensembles of side effects, $P$,
  $Q$, $R$ are asserts, such that $\local{E_2}{R}$, and $t$ is a side
  effect that belongs to $E_1$, then from and $\{P\} [t] \{Q\}$
  follows $\{P \land R\}[t]\{Q \land R\}$.
\end{theorem}
\begin{proof}
  See \mbox{frame\_rule}
\end{proof}

\begin{definition}
  Chain rule $\kappa$ is local under ensemble of side effects $E$ when

  \begin{equation}
    \crlocal{E}{\kappa} \triangleq \forall t_1\, t_2, t_1 \in E \land t_2 \notin E \to \comm{t_1}{t_2}
  \end{equation}
\end{definition}


\section{Handlers}

\section{Processes}

SLOT is used to define an experimental framework for modeling
functional programs with side effects, implemented as a shallow
embedded DSL in Coq.

\end{document}

\usepackage{amsmath}
